\documentclass[12pt,a4paper]{article}
%%%%%%%%%%%%%%%%%%%%导言区%%%%%%%%%%%%%%%
\usepackage[BoldFont,SlantFont]{xeCJK}
\setCJKmainfont[BoldFont=SimSun,ItalicFont=SimSun]{SimSun}
\setmainfont{Times New Roman}

\title{NiosII Linux Tutorials}
\author{BearChen}
\date{2010.11.29}

%%%%%%%%%%%%%%%%%%%%正文%%%%%%%%%%%%%%%
\begin{document}
\maketitle{}
\newpage{}
%%%%%%%%%%%%%%%%%%%%前言%%%%%%%%%%%%%%%
\section{前言}
\noindent{}本文由\LaTeX{}和xe\LaTeX{}构建.\\这是一个\textbf{NiosII Linux}开发的入门教程,拟包括以下几个部分:\footnote{随着编写计划的变化,可能会有增减.}
\begin{enumerate}
\item 介绍\textbf{NiosII Linux}开发环境;
\item 一个完整的流程;
\item 创建自己的SOPC系统和Linux内核;
\item linux下驱动开发举例;
\item linux下普通应用程序开发;
\item FTK应用程序开发;
\end{enumerate}
%%%%%%%%%%%%%%%%%%%%构建开发环境%%%%%%%%%%%%%%%
\newpage{}
\section{构建开发环境}

\end{document}
