\documentclass[12pt,a4paper,titlepage]{article}
%%%%%%%%%%%%%%%%%%%%导言区%%%%%%%%%%%%%%%
%%宏包设置
\usepackage[BoldFont]{xeCJK}
\usepackage[dvips]{graphicx}
\usepackage[colorlinks,linkcolor=red,anchorcolor=red,citecolor=red,urlcolor=blue]{hyperref}
%\usepackage{nl}
%%字体设置
\setCJKmainfont[BoldFont=SimSun,ItalicFont=SimSun]{SimSun}
\setmainfont{Times New Roman}
\setCJKfamilyfont{ht}{SimHei}
%%页面设置
\addtolength{\hoffset}{-1cm}
\addtolength{\voffset}{-2cm}
\addtolength{\textheight}{4cm}
\addtolength{\textwidth}{2cm}
%%标题页
\title{NIOS II Linux Tutorial}
\author{作者:陈锋\\\\审校:邢岚\\\\Email:cfistalent@gmail.com\\\\重庆大学通信工程学院2401学习资料}
%%%%%%%%%%%%%%%%%%%%正文%%%%%%%%%%%%%%%
\begin{document}
\thispagestyle{empty}
\maketitle{}
%%%%%%%%%%%%%%%%%%%%前言%%%%%%%%%%%%%%%
\section{前言}
本文由\LaTeX{}构建.
%源代码在\href{https://github.com/cfistalent/NiosII\_Linux\_Tutorial}{https://github.com/cfistalent/NiosII\_Linux\_Tutorial}上.
%可通过\fbox{git clone git://github.com/cfistalent/NiosII\_Linux\_Tutorial.git}获取最新的源码.

这是一个\textbf{NIOS II Linux}开发的入门教程,拟包括以下几个部分:\footnote{随着编写计划的变化,可能会有增减.}
\begin{itemize}
\item 如何获取资源;
\item 介绍\textbf{NIOS II Linux}开发环境;
\item 一个完整的流程,创建硬件到配置内核;
\item 开发前需要了解的知识;
\item linux下普通应用程序开发;
\item linux下驱动开发举例;
\item FTK应用程序开发;
\end{itemize}
%%%%%%%%%%%%%%%%%%%%如何获得相关资料%%%%%%%%%%%%%%%
\newpage{}
\section{如何获得相关资料}
\subsection{网站}
\begin{description}
\item[\href{http://www.nioswiki.com/Linux}{NiosWiki上Linux主页}] 这是nios linux内核开发包的发布主页.
上面包含如何下载内核开发包,如何进行第一次编译等工作.有相应文档提供下载.
\item[\href{http://www.nioswiki.com/OperatingSystems/UClinux}{NiosWiki上uClinux主页}] 这里包含的内容大多数都适用于Linux.
因此,在动手之前请把这两个页面仔细读一读.
\item[\href{http://www.alteraforum.com/forum/forumdisplay.php?f=37}{Altera论坛}] 这里提供了一个问题交流的地方,学习过程中
遇到的问题都可以在这里获得响应.不过要注意两点:{\CJKfamily{ht}首先确认你在wiki上找不到答案,其次有合格的英语表达能力}.
\item[\href{http://sopc.et.ntust.edu.tw/pipermail/nios2-dev/}{nios linux开发者邮件列表}] 这里是nios linux的真正发源地.每个月
都会有若干个bug在这里被修复,新的版本的内核被移植.时刻关注这里可以让你与开发者同步.
\item[\href{ftp://ftp.altera.com/up/pub/}{Altera官方FTP}] 这里面资源很丰富,Altera FPGA的开发资料,Altera大学IP等等.
\item[\href{http://www.cnblogs.com/oomusou/}{一个台湾工程师的博客}] FPGA,特别是DE2和DE2-70的资源很丰富.
\item[\href{http://www.terasic.com/downloads/cd-rom/}{友晶公司的CD资源FTP}] 包含友晶公司各种开发板附带的CD资料.需要在友晶官网注册.
\end{description}
\subsection{书籍}
\begin{description}
\item[\href{http://www.amazon.cn/mn/detailApp/ref=sr\_1\_1?\_encoding=UTF8&s=books&qid=1291516849&asin=B001EJNTCC&sr=8-1}{SOPC嵌入式系统基础教程}] 介绍SOPC基础知识.是Altera官方文档的中文翻译版.有能力的同学可以直接看\href{http://www.altera.com.cn/literature/lit-nio2.jsp}{官方文档}.
\item[\href{http://www.amazon.cn/Nios-2\%E8\%BD\%AF\%E4\%BB\%B6\%E6\%9E\%B6\%E6\%9E\%84\%E8\%A7\%A3\%E6\%9E\%90-\%E8\%94\%A1\%E4\%BC\%9F\%E7\%BA\%B2/dp/B0011Y1VXO/ref=pd\_sim\_b\_2}{Nios2软件架构解析}] 将Altera的Nios2软件架构分析的很透彻,尽管针对的是6.0版本依然很有用.
\item[\href{http://www.amazon.cn/mn/detailApp/ref=sr\_1\_1?\_encoding=UTF8&s=books&qid=1291517188&asin=B004A7L87I&sr=1-1}
{Linux设备驱动开发详解}] 宋宝华老师的书,针对Linux驱动开发做了详尽的介绍.
\item[\href{http://www.amazon.cn/mn/detailApp/ref=sr\_1\_1?\_encoding=UTF8&s=books&qid=1291517330&asin=B003Q97NPC&sr=1-1}{Linux程序设计}]
 介绍Linux应用程序开发的好书.
\end{description}
%%%%%%%%%%%%%%%%%%%%构建开发环境%%%%%%%%%%%%%%%
\newpage{}
\section{构建开发环境}
2009年9月,nioswiki社区推出基于NIOS II处理器的linux内核开发包,为开发者提供一种新的系统方案.

使用\textbf{NIOS II Linux}开发需要搭建硬件和软件环境.目前已知的开发环境有两类:
\begin{table}[!hbtp]
\centering
\begin{tabular}{|l|p{0.68\textwidth}|}
\hline
完全在linux下开发 & 在linux下安装linux版本的Quartus II和NiosII IDE开发工具,同时安装nios-linux内核开发包.
据我所知目前只有\textbf{Ubuntu9.04}可以兼容,Ubuntu9.10及以上版本暂时不支持.\footnote{不支持的原因是nios2-download脚本在Ubuntu9.10及
以上版本中部分无法执行,而官方推荐的Redhat企业版和Fedora等均可配置}\\
\hline
WindowsXP+Linux虚拟机 & 这种方案更适合大多数平时工作在Windows环境下的开发者.本文接下来将介绍这种方案的配置.\footnote{由于在linux下
的usb-blaster驱动比windows下性能要相对差些,当构建的系统选择以JTAG-UART作为终端时,在linux环境下的"nios2-terminal"命令会导致系统阻塞,此时
我更倾向于使用本方案}\\
\hline
\end{tabular}
\caption{开发环境\label{dev_env}}
\end{table}
%%%3.1
\subsection{硬件环境}
首先需要在Windows上安装Quartus II和相应版本的NiosII IDE.本文使用是9.0版本.因此大于等于该版本都应该没有问题.
安装过程这里不赘述,有问题请自行查阅相关文档.但是安装的版本{\CJKfamily{ht}需要有lisence支持}.
%%%3.2
\subsection{软件环境}
linux内核驱动和应用程序的开发需要在类Unix操作系统下进行,最普遍的就是Linux操作系统.这里我们选择流行程度最高的Ubuntu操作系统作为开发平台.

首先我们要安装一款虚拟机软件\footnote{Windows下有Vmware,Virtual Box等}.通过该软件创建虚拟机并安装Ubuntu.安装好的虚拟机系统要满足两个要求:
\begin{itemize}
\item 能够连接互联网,下载内核开发所需的文件;
\item 能够与宿主机(Windows)交换文件.宿主机下生成的头文件要传入虚拟机中交给内核开发包使用,而虚拟机下生成的内核可执行文件要
在window下通过nios shell下载;
\end{itemize}

我们以Virtual Box为例说明如何操作\footnote{使用Vmware等只要满足上面两个条件同样可以搭建开发环境}.
%%%3.2.1
\subsubsection{虚拟机安装Ubuntu并做相关设置}
创建虚拟机之前,本地电脑上需要准备如下内容:
\begin{itemize}
\item Ubuntu安装光盘或ISO镜像文件;
\item 已安装好Virtual Box软件;
\item 足够的硬盘空间(14G左右);
\end{itemize}

在Virtual Box主页面菜单上选择\fbox{控制-->新建},操作系统类型选择\textbf{Linux},版本选择\textbf{Ubuntu},如图\ref{f_create_vb1}所示.
\begin{figure}[!bthp]
\centering\includegraphics[width=1\textwidth]{pic/f_create_vb1.eps}
\caption{创建虚拟机:整体设定\label{f_create_vb1}}
\end{figure}
内存选择为当前系统内存一半略少,尽量发挥系统性能.硬盘选择"创建新的虚拟硬盘"并选择"固定大小"\footnote{相对于动态增长,
固定大小使得虚拟机运行速度更快}.硬盘大小至少10G,以免内核开发包在虚拟机中解压缩后空间不够,{\CJKfamily{ht}注意,默认虚拟硬盘保存在系统盘中,
请将位置改到有足够空间的磁盘上}.硬盘创建完成后确定即可.

到现在为止,虚拟机已经创建完毕.但是当前的虚拟机内没有任何系统,甚至连分区都没有,此时需要通过光驱安装操作系统.本文使用ISO镜像
文件进行安装.

在没有运行虚拟机的情况下,点击\fbox{设置}图标,右侧找到"介质"一栏,对ISO进行注册.如图\ref{iso0},\ref{iso1}和\ref{iso2}所示.
\begin{figure}[!bthp]
\centering
\includegraphics[width=0.8\textwidth,scale=0.8]{pic/f_vb_setting_iso.eps}
\caption{设置ISO\label{iso0}}
\includegraphics[width=0.8\textwidth,scale=0.8]{pic/f_vb_setting_iso_register.eps}
\caption{设置ISO\label{iso1}}
\end{figure}
\begin{figure}[!bthp]
\centering
\includegraphics[width=0.8\textwidth,scale=0.8]{pic/f_vb_setting_iso_set.eps}
\caption{设置ISO\label{iso2}}
\end{figure}

之后运行虚拟机,系统就会从ISO进行引导,进而安装Ubuntu系统.过程略.

安装完成后.若宿主机能够访问互联网,则虚拟机也能访问互联网.剩下一个问题就是共享文件夹的设置.在设置之前,需要先安装Virtual Box提供的
增强工具包.在运行虚拟机的界面上选择\fbox{设备-->安装增强功能}.此时在虚拟机的桌面上就会出现一个光盘.打开终端\footnote{Alt+F2调出运行
窗口,输入\fbox{gnome-terminal}回车即可调出终端}.在终端中输入如下命令\footnote{下文中没有特别说明,运行的命令都将是在终端中}:
\begin{verse}
cd /media/cdrom\\sudo ./VBoxLinuxAdditions-x86.run \#若本地PC是amd64位架构,就运行VBoxLinuxAdditions-amd64.run\\输入管理员密码
\end{verse}
等待安装过程结束,关闭虚拟机{\CJKfamily{ht}(注意,这里一定要先关闭虚拟机)}.然后在\fbox{设置}中右侧找到"数据空间",添加一个在宿主机上已经存在的文件夹作为共享目的地.如图\ref{f_sf}所示.
启动虚拟机,终端中输入如下命令:
\begin{verse}
mkdir -p /home/<你的用户名>/Desktop/VBS \#在虚拟机中创建一个文件夹\\sudo mount -t vbox <宿主机中共享文件夹名称> 
/home/<你的用户名>/Desktop/VBS
\end{verse}
经过这番设置后,共享文件夹就设置成功了.但是,重启后设置将会失效,为使启动后系统自动处理共享文件夹,需进行如下配置.
\begin{verse}\label{cmd0}
sudo gedit /etc/fstab\\在打开的文件中添加一行" VBoxSharedFolder  /home/<你的用户名>/Desktop/VBS/  vboxsf  defaults 0 0"
\end{verse}
\begin{figure}[!bthp]
\centering
\includegraphics[width=0.8\textwidth,scale=0.8]{pic/f_vb_setting_sf.eps}
\caption{共享空间设置\label{f_sf}}
\end{figure}
%%%3.2.2
\subsubsection{内核开发包安装设置}
在安装内核开发包之前,要先安装一些基本的软件包:
\begin{verse}
sudo apt-get install git-core git-doc automake libtool enca ncurses-dev pax-utils
\end{verse}

将我提供的内核开发包\footnote{nios2-linux\_CF.tgz}拷贝到{\CJKfamily{ht}宿主机}的共享文件夹下,在{\CJKfamily{ht}虚拟机终端}运行如下命令:
\begin{verse}
cd /\\sudo chown <你的用户名,在这里,是bearchen:/> opt/\\mkdir NiosIILinux\\cd /opt/NiosIILinux/\\
tar -zxvf /home/<你的用户名>/Desktop/VBS/<内核开发包名称>
\end{verse}
我们将把内核开发不包安装在/opt目录下,前两个命令的目的是修改opt的所有者,方便之后运行命令.第三个命令则是我的内核开发包历史遗留问题.
第四个命令将内核解压缩到/opt/NiosIILinux/目录下.

{\CJKfamily{ht}注意,接下来的操作比起从官方网站下载的开发包要少一道步骤,就是将源码从git仓库中导出.因为我提供的源码包已经导出相关源码.}
接着安装内核开发所需的工具,包括gcc编译器等(其实就是将工具程序所在文件夹注册到系统环境变量中):
\begin{verse}
gedit ~/.bashrc\\在打开的文件中最后添加一行"export PATH=\$PATH:/opt/NiosIILinux/nios2-linux/toolchain-mmu/x86-linux2/bin"保存修改\\
source ~/.bashrc \#更新环境变量
\end{verse}
通过以上操作,内核开发的软件环境已经搭建完毕.为了测试当前是否搭建成功.可做如下操作:
\begin{enumerate}
\item 运行以下命令:
\begin{verse}
cd /opt/NiosIILinux/nios2-linux/uClinux-dist/\\make clean\\make menuconfig
\end{verse}
在打开的界面中\footnote{这是内核的配置菜单,在这个菜单中,Enter表示进入下一层,ESC两次表示退回上一层或者退出,SPACE表示修改选项状态}
选择"Kernel/Library/Defaults Selection--->"进入.选中"Default all settings(lose changes)".退出保存.对遇到的所有提问选择'N'.
\item 再次运行\fbox{make menuconfig}.进入"Kernel/Library/Defaults Selection--->".选中"Customize Kernel Settings".退出保存.
\item 在新弹出的配置窗口,进入"Device driver--->Network device support--->Ethernet(10 or 100mbit)--->"将"Altera Triple Speed Ethernet 
MAC Support(SLS)"选项取消.退出保存.
\item 运行以下命令\footnote{rc是linux内核启动后会自动执行的脚本文件,
可以通过他人为设置开机后自动运行的命令序列.这里设置为空,避免之后编译出的内核运行时出现不必要的现象.}:
\begin{verse}cd /opt/NiosIILinux/nios2-linux/uClinux-dist/vendors/Altera/common/\\echo > rc\end{verse}
\item 运行\fbox{make}.若接下来的过程没有错误.恭喜你.整个环境已经配置成功.
\end{enumerate}
%%%3.3
\subsection{体验NIOS II Linux}
如果你手头上正好有一块DE2-70开发板,则可以通过以下方式体验一下linux在nios处理器上运行的感觉.
\begin{enumerate}
\item 找到本文附带的文件夹中的两个文件"Starter\_Hardware.sof"和"Starter\_Kernel.gz".将他们拷贝到同一个文件夹下.
\item 运行nios shell.如图\ref{ns}所示.
\item 将DE2-70的电源和usb-blaster连接好.上电.
\item 在nios shell下进入刚才保存两个文件的目录.{\CJKfamily{ht}运行命令}\footnote{注意,这里是在宿主机的nios shell下输入命令}:
\begin{verse}
nios2-config-sof Starter\_Hardware.sof\ {}\ {}\&\&\ {}\ {}nios2-download -g Starter\_Kernel.gz\ {}\ {}\&\&\ {}\ {}nios2-terminal
\end{verse}
在nios shell上打印一堆信息后将看到类似终端的命令提示符.
若一切正常,此时的nios shell将作为开发板上运行的linux内核的终端与用户交互,你可以尝试输入命令获得响应.
\end{enumerate}
\begin{figure}[!htbp]
\centering
\includegraphics[width=0.75\textwidth,scale=0.75]{pic/f_shell_and_icon.eps}
\caption{Nios Shell图标和运行界面\label{ns}}
\end{figure}
%%%%%%%%%%%%%%%%%%%%走一遍流程%%%%%%%%%%%%%%%
\newpage{}
\section{第一个工程}
阅读本节内容的前提是读者已了解Quartus II和SOPC Builder的使用方法.否则请自行学习相关内容.

本节通过一个简单工程介绍整个设计流程.使用的硬件是\textit{DE2-70}\footnote{使用其他开发板一样可以完成本节内容,只需要
满足NiosII Linux开发的硬件最低要求}.整个过程包括但不限于:{\CJKfamily{ht}SOPC构建,顶层模块设计及管脚分配,linux内核
简单配置,下载运行}.
%%%4.1
\subsection{硬件设计}
构建的片上系统必须满足以下几个硬性条件,才能够运行linux内核和应用程序:
\begin{itemize}
\item FPGA芯片至少是Cyclone\footnote{这是基于当前开发社区提供资料的推测};
\item cpu类型至少是标准型(s),快速型(f)更佳;
\item 内存至少为8MB;
\item 必须有一个全功能定时器外设(timer),定时时间为10ms;
\item 必须有一个linux内核与外界交互的终端.一般为JTAG-UART或者UART;
\end{itemize}

\begin{enumerate}
\item 使用Quartus II的工程创建向导新建一个工程.启动SOPC Builder\footnote{{\CJKfamily{ht}接下来所有组件的设置,对没有提到的参数,
一律为默认设置}},添加CPU,{\CJKfamily{ht}无视错误信息},先保存.
添加一个SDRAM Controller.存储器设置内容\footnote{相关信息根据\textit{DE2-70}开发板上SDRAM
参数填写.其他开发板请查阅相应SDRAM数据手册}:
\begin{table}[!bhtp]
\centering
\begin{tabular}{|c|c|c|c|c|c|}
\hline
Presets & Data Width & Chip select & Banks & Row & Column \\
\hline
Custom & 16 & 1 & 4 & 13 & 9\\
\hline
\end{tabular}
\caption{SDRAM Controller:Memory Profile}
\end{table}
\begin{table}[!bhtp]
\centering
\begin{tabular}{|l|c|}\hline
CAS latency cycles & 3\\\hline
Initialization refresh cycles & 2\\\hline
Issue one refresh command every & 7.8125 us\\\hline
Delay after powerup,before initialization & 200us\\\hline
Duration of refresh command(t\_rfc) & 70ns\\\hline
Duration of precharge command(t\_rp) & 20ns\\\hline
ACTIVE to READ or WRITE delay(t\_rcd) & 20ns\\\hline
Access time(t\_ac) & 5.5ns\\\hline
Write recovery time(t\_wr,no auto precharge) & 14ns\\\hline
\end{tabular}
\caption{SDRAM Controller:Timing}
\end{table}
\\保存为默认名称\textbf{"sdram\_0"}.
\item 为了使用CPU的MMU功能,需要添加一个onchip memory,需要配置的内容:类型为\textbf{RAM},\textbf{Dual-port access},
内存大小\textbf{1024 Bytes}.其他配置默认.名称没有指定,只要有意义即可.
\item 打开CPU配置界面,Reset Vector和Exception Vector都选为\textbf{sdram\_0}.Exception Vector的偏移地址选为\textbf{0x20}.
钩选\textbf{"Include MMU"},Fast TLB Exception Vector选为刚才添加的onchip memory.偏移量为\textbf{0}.在CPU的"Caches and Memory interfaces"
选项卡中,钩选\textbf{"Include tightly coupled instruction master port(s)"}和
\textbf{"Include tightly coupled data master port(s)"}.在"JTAG Debug Module"选项卡中,
将JTAG选为\textbf{level 2}.其余选项均默认.确定退出CPU设置.对cpu和sdram controller以及onchip memory
作图\ref{f_cpu_sdram_tcm}方式连接.
\begin{figure}[!bhtp]
\centering
\includegraphics[width=0.6\textwidth]{pic/f_sopc_cpu_tcm_sdram.eps}
\caption{CPU与SDRAM Controller以及Fast FLB连接方式\label{f_cpu_sdram_tcm}}
\end{figure}
\item 添加一个Interval Timer作为linux内核的系统时钟.周期设为\textbf{10 ms},计数器位宽\textbf{32 bit},Presets为\textbf{Full-featured}.
\underline{{\CJKfamily{ht}timer的中断优先级一定设置为0}}.
\item 添加一个JTAG-UART,默认设置.
\item 添加一个UART,默认设置.
\item 以上构成了linux运行所需的最小配置(严格来说,JTAG-UART和UART可以只选一个,但必须有).\textit{DE2-70}上晶振为50Mhz,
为提高系统运行速度,我们添加一个PLL,运行ALTPLL MegaWizard,启用两个输出c0和c1.将c0输出为输入的一倍(100Mhz),其余值默认.
将c1输出也设置为输入的一倍(100Mhz).唯一不同的是,c1的Clock phase shift设置为\textbf{-65 deg}.
\item 修改SOPC系统中每一个组件的时钟,除了PLL是默认的外部输入时钟外,其余均改为pll\_c0时钟.
\item 点击System菜单下的Auto-Assign Base Address自动分配各组件基址以消除地址冲突.注意Auto-Assign IRQs这个选项很可能会将timer的优先级
修改,如果使用,请一定小心.完成后SOPC Builder界面如图\ref{f_sopc}所示.保存当前设置.点击Generate生成SOPC系统.
\begin{figure}[!bthp]
\centering
\includegraphics[width=1\textwidth]{pic/f_sopc.eps}
\caption{完成设置的SOPC\label{f_sopc}}
\end{figure}
\item 在Quartus II下新建一个.v文件作为顶层文件.添加如下内容:
\begin{verse}
module NL\footnote{顶层模块名称,保存.v文件时文件名必须与该名称一致} (\\
  input         iCLK\_50,        // 50 MHz\\
  //////////////////////////////// UART ////////////////////////////\\
  output        oUART\_TXD,      // UART Transmitter\\
  input         iUART\_RXD,      // UART Receiver\\
  //////////////////////////////// SDRAM Interface ////////////////////////\\
  inout  [31:0] DRAM\_DQ,        // SDRAM Data bus 32 Bits\\
  output [12:0] oDRAM0\_A,       // SDRAM0 Address bus 12 Bits\\
  output        oDRAM0\_LDQM0,   // SDRAM0 Low-byte Data Mask \\
  output        oDRAM0\_UDQM1,   // SDRAM0 High-byte Data Mask\\
  output        oDRAM0\_WE\_N,    // SDRAM0 Write Enable\\
  output        oDRAM0\_CAS\_N,   // SDRAM0 Column Address Strobe\\
  output        oDRAM0\_RAS\_N,   // SDRAM0 Row Address Strobe\\
  output        oDRAM0\_CS\_N,    // SDRAM0 Chip Select\\
  output [1:0]  oDRAM0\_BA,      // SDRAM0 Bank Address\\
  output        oDRAM0\_CLK,     // SDRAM0 Clock\\
  output        oDRAM0\_CKE      // SDRAM0 Clock Enable\\
);\\
// NIOS CPU\\
wire CPU\_CLK;\\
wire SYS\_RESET\_N;\\
assign SYS\_RESET\_N=1'b1;\\
// NIOS II system\\
Nios\footnote{这是已构建的SOPC系统的名称} nios2 (\\
.clk\_in(iCLK\_50),\\
.pll\_system(CPU\_CLK),\\
.pll\_sdram(oDRAM0\_CLK),\\
.reset\_n(SYS\_RESET\_N),\\
// the\_sdram (u1)\\
.zs\_addr\_from\_the\_sdram\_0(oDRAM0\_A),\\
.zs\_ba\_from\_the\_sdram\_0(oDRAM0\_BA),\\
.zs\_cas\_n\_from\_the\_sdram\_0(oDRAM0\_CAS\_N),\\
.zs\_cke\_from\_the\_sdram\_0(oDRAM0\_CKE),\\
.zs\_cs\_n\_from\_the\_sdram\_0(oDRAM0\_CS\_N),\\
.zs\_dq\_to\_and\_from\_the\_sdram\_0(DRAM\_DQ),\\
.zs\_dqm\_from\_the\_sdram\_0({oDRAM0\_UDQM1,oDRAM0\_LDQM0}),\\
.zs\_ras\_n\_from\_the\_sdram\_0(oDRAM0\_RAS\_N),\\
.zs\_we\_n\_from\_the\_sdram\_0(oDRAM0\_WE\_N),\\
// the\_uart\_0\\
.rxd\_to\_the\_uart\_0(iUART\_RXD),\\
.txd\_from\_the\_uart\_0(oUART\_TXD)\\
);\\
endmodule\\
\end{verse}
将该文件设置为顶层文件.进行编译.编译结束后,查询\textit{DE2-70}数据手册,分配顶层各信号管脚.再进行一次编译.整个硬件系统就搭建完毕.
\end{enumerate}
%%%4.2
\subsection{内核配置和编译}
现在我们来配置当前SOPC系统对应的linux内核.
\begin{enumerate}
\item 打开nios shell(如图\ref{ns}所示),进入到硬件工程所在目录下,运行:
\begin{verse}
sopc-create-header-files --single custom\_fpga.h\footnote{生成的头文件名字一定为\textbf{custom\_fpga.h}}
\end{verse}
这将会在该目录下生成名为\textbf{custom\_fpga.h}的C语言头文件.将这个文件拷贝到共享文件夹当中.
\item 启动虚拟机,输入如下命令:
\begin{verse}
cp -i /home/bearchen\footnote{这是我的Ubuntu虚拟机下用户名,请替换成你自己的用户名}/Desktop/VBS\footnote{这是我的共享文件夹名称,同样
请替换成你的}/custom\_fpga.h /opt/NiosIILinux/nios2-linux/linux-2.6/arch/nios2/include/asm/\\
确认替换当前已有文件
\end{verse}
\item 输入命令:
\begin{verse}
cd /opt/NiosIILinux/nios2-linux/uClinux-dist/\\
make menuconfig\\
\end{verse}
进入内核配置界面.由于本系统比较简单,除了JTAG-UART和UART外没有任何外设,因此内核中设备驱动部分只需设置两种串口驱动支持以及
哪一种设备作为终端输出.但是,有一些设置虽然只需配置一次,但影响全局,需要特别注意.为介绍整个过程,我们先将当前配置除去.
选择"Kernel/Library/Defaults Selection--->"进入.选中"Default all settings(lose changes)".退出保存.对遇到的所有提问选择'N'.将整个内核包
恢复到默认发布配置.
\item 再次运行\fbox{make menuconfig}.进入"Kernel/Library/Defaults Selection--->".选中"Customize Kernel Settings".退出保存.
\item 在新弹出的配置窗口,进入"NiosII Configuration--->",将"NiosII FPGA Configuration(MMU\_DEFAULT)"修改为"NiosII FPGA Configuration
(CUTOM\_FPGA)"
\item 在同一个窗口中将"Link address offset fot booting"的值改大,防止内核解压缩后的代码覆盖压缩内容,这里改为\textbf{0x00F00000}.
\item 退出到上一个界面,进入"Device driver--->Network device support--->Ethernet(10 or 100mbit)--->"将"Altera Triple Speed Ethernet 
MAC Support(SLS)"选项取消\footnote{开发包默认使用Stratix的开发板和这一款网络驱动,与本工程不符.若不取消,编译报错.}.退出保存.
\item 退回到"Device driver--->"下,进入"Character devices--->Serial drivers".选择JTAG-UART和UART驱动支持,并开启JTAG-UART终端(console)支持.
如图\ref{f_kcfg_ct}所示.设置完成,退出并保存.
\begin{figure}[!bhtp]
\centering
\includegraphics[width=0.6\textwidth]{pic/f_kcfg_char_terminal.eps}
\caption{内核配置:终端配置\label{f_kcfg_ct}}
\end{figure}
\item 运行\fbox{make}.内核编译完成.将images/zImage.initramfs.gz拷贝到共享文件夹\footnote{{\CJKfamily{ht}共享文件夹是联系宿主机和
虚拟机的通道,在宿主机和虚拟机中都有一个相应的文件夹进行关联}}中.
\end{enumerate}
%%%4.3
\subsection{下载运行}
此时在宿主机的共享文件夹中就存在有从虚拟机拷贝出来的内核镜像文件,为了让linux在开发板上跑起来,需要按顺序下载{\CJKfamily{ht}硬件sof文件,
内核镜像文件}.

下载sof文件有两种方法,一种是在Quartus II中直接调用Programmer下载,另一种通过nios shell下调用命令来进行.这里采用第二种.将硬件sof文件和
内核文件拷贝到同一个文件夹下.连接好\textit{DE2-70}开发板的下载线和电源,上电,在nios shell下进入该文件夹,运行如下命令:
\begin{verse}
nios2-config-sof <sof文件> \&\& nios2-download -g <内核镜像文件> \&\& nios2-terminal
\end{verse}
下载完成后,将会看到大量的打印信息,最终在shell上会出现类似Ubuntu下终端提示符的界面,如图\ref{f_kb}所示.你可以输入ls命令查看当前开发板上的
文件系统,进入/bin目录运行可执行文件等.当你执行这些操作时,恭喜你,linux已经在开发板上跑起来了.
\begin{figure}[!bhtp]
\centering
\includegraphics[width=0.8\textwidth]{pic/f_kernel_bootingmsg.eps}
\caption{内核启动完成\label{f_kb}}
\end{figure}

提醒:对于入门的同学,掌握整个开发流程是必要的;而对于进行大规模系统开发的同学,建议使用模板进行修改,效率会得到很大的提高.
%%%%%%%%%%%%%%%%%%%%开发前需要了解的知识%%%%%%%%%%%%%%%
\newpage{}
\section{开发前需要了解的知识}
本节将介绍一些开发所需要的背景知识,但不会进行深入的讨论.一来笔者本身也只留于表面;二来有违本文宗旨,且篇幅不允许.

通过上一节的例子,读者应该对NIOS II Linux系统开发的基本步骤有了大概的认识.至于每一步的目的是什么恐怕会有许多迷惑.本节将对内核开发中的若干
问题进行解释.纯属个人理解,水平有限,如有不对的地方,欢迎同学与我\href{mailto:cfistalent@gmail.com}{联系}.
%%%5.1
\subsection{内核开发包目录结构}
解压完内核开发包,生成名为"nios-linux"的文件夹,进入该文件夹,可以看到有如下文件和文件夹:
\begin{quote}
3c120\_default binutils checkout elf2flt gcc3 glibc insight\\
linux-2.6 README sshkey toolchain-build toolchain-mmu\\
u-boot uClibc uClinux-dist update use\_git\_for\_update\\
use\_http\_for\_update use\_ssh443\_for\_update
\end{quote}
其中有几个比较重要:
\begin{description}
\item[\textit{linux-2.6}] 保存了linux内核的所有相关代码,其中大部分是各种驱动程序.驱动程序移植和开发工作都将在这个目录下
的driver目录中进行.同时arch/nios目录下保存有nios处理器相关的底层代码,包括各种片上系统对应的头文件,内核启动代码,各种片上外设的配置等.
\item[toolchain-mmu] 存有nios linux开发工具集,包括gcc,gdb等.
\item[u-boot] 存有引导linux操作系统的著名的bootloader.
\item[glibc] nios linux下开发应用程序所采用的标准C语言库函数.
\item[\textit{uClinux-dist}] 这是开发时大部分时间都将要驻留的文件夹,在这个文件夹中,有nios linux下应用程序开发时使用的各种库函数
,也有已经移植到nios linux下的各种应用程序,二者都以源码的形式存放在这个文件夹的lib和user文件夹中.
同时,这里也是运行内核配置,编译内核等命令的目录.随后我会作进一步介绍.
\item[uClibc] 内核开发包中其实包含了两个操作系统源码,带MMU的nios处理器支持的linux和不带MMU的nios处理器支持的uClinux.而uClibc就是
uClinux系统下C语言的标准函数库.两个操作系统源码存在于git\footnote{Linus开发的一个分布式版本管理系统}的两个不同分支中,
通过\fbox{git checkout}命令切换.
\item[checkout] 这是一个shell脚本,通过该脚本可利用git更新本地各个文件夹源码的版本.
\end{description}

%%%5.2
\subsection{三个重要的文件}
%%%5.3
\subsection{内核开发中使用的部分工具介绍}
%%%5.4
\subsection{如何使用内核自带的设备驱动}
\end{document}
